% colors from Color Brewer 2.0, Set1: http://colorbrewer2.org/#type=qualitative&scheme=Set1&n=3
\definecolor{parameter}{HTML}{e41a1c}
\definecolor{result}{HTML}{377eb8}
\definecolor{sort}{HTML}{4daf4a}

% colors from Color Brewer 2.0, Pastel1: http://colorbrewer2.org/#type=qualitative&scheme=Pastel1&n=9
\definecolor{IC}{HTML}{ffffcc}
\definecolor{IS}{HTML}{fed9a6}
\definecolor{INS}{HTML}{e5d8bd}
\definecolor{BI}{HTML}{decbe4}
\definecolor{DEL}{HTML}{33ccff}

\newboolean{standalone}

\reversemarginpar
\newcommand{\currentQueryCard}{0}
\newcommand{\queryRefCard}[3]{
	\ifthenelse{
		\equal{\currentQueryCard}{#3}
	}{%
		\colorbox{white}{\tt #2 #3}%
	}{%
		\ifthenelse{
			\boolean{standalone}
		}{%
			\href{https://www.ldbcouncil.org/ldbc_snb_docs/#1.pdf}{\colorbox{#2}{\tt #2 #3}}%
			%\colorbox{#2}{\tt #2 #3}%
		}{%
			\hyperref[sec:#1]{\colorbox{#2}{\tt #2 #3}}%
		}
	}%
}

\newcommand{\attributeNumberWidth}{0.33cm}
\newcommand{\attributeColumnWidth}{2.5cm}
\newcommand{\typeColumnWidth}{2.7cm}
\newcommand{\descriptionColumnWidth}{10.3cm}
\newcommand{\largeDescriptionColumnWidth}{13cm}

\newcommand{\tableHeaderFirst}[1]{\multicolumn{1}{|c|}{\bf #1}}
\newcommand{\tableHeader}[1]{\multicolumn{1}{c|}{\bf #1}}

% using camelCase notation is not conventional in LaTeX, but it helps readability a lot, so I decided to use it anyways [szarnyasg]

\newcommand{\queryCardWidth}{17cm}
\newcommand{\queryPropertyCell}{\small \sf \centering}
\newcommand{\queryPropertyCellWidth}{1.48cm}

\newcommand{\attributeCardWidth}{14.66cm}
\newcommand{\typeWidth}{2.04cm}

\newcommand{\paramNumberCell}{\cellcolor{parameter}\color{white}\footnotesize}
\newcommand{\resultNumberCell}{\cellcolor{result}\color{white}\footnotesize}
\newcommand{\sortNumberCell}{\cellcolor{sort}\color{white}\footnotesize}

\newcommand{\directionCell}{\cellcolor{gray!20}}
\newcommand{\resultOriginCell}{\tt}
\newcommand{\edgeDirectionCell}{\tt}

% for hyphenating tt text, see also https://tex.stackexchange.com/a/44362/71109
\newcommand{\varNameText}{\tt}
\newcommand{\varNameCell}{\varNameText}

\newcommand{\typeText}{\footnotesize\sf}
\newcommand{\typeCellBase}{\cellcolor{gray!20}\typeText}
\newcommand{\typeCell}{\typeCellBase\raggedright}

\newcommand{\chokePoint}[1]{\hyperref[choke_point_#1]{#1}}

\newcommand{\innerCardVSpace}{\vspace{1.1ex}}
\newcommand{\queryCardVSpace}{\vspace{2ex}}

% tabularx magic
% https://tex.stackexchange.com/a/89932/71109
\newcolumntype{Y}{>{\raggedright\arraybackslash}X}
% https://tex.stackexchange.com/questions/252385/mixing-m-and-x-in-tabularx#comment602205_252388
\renewcommand{\tabularxcolumn}[1]{m{#1}}

%\newcolumntype{R}[1]{>{\raggedleft\let\newline\\\arraybackslash\hspace{0pt}}m{#1}}
\newcolumntype{C}[1]{>{\centering\let\newline\\\arraybackslash\hspace{0pt}}m{#1}}


% https://tex.stackexchange.com/a/385069/71109
\setlength\cellspacetoplimit{3pt}
\setlength\cellspacebottomlimit{3pt}
\newcolumntype{M}{>{\begin{varwidth}{3.8cm}}Sl<{\end{varwidth}}}

\newcommand{\attributeTable}[3]{
	\vspace{1ex}
	\begin{tabularx}{\linewidth}{|l|Y|}
	\hline
	\bf Attribute   & \varNameCell  #1 \\ \hline
	\bf Type        & \typeCellBase #2 \\ \hline % using typeCellBase as we cannot use \raggedright here
	\bf Description &               #3 \\
	\hline
	\end{tabularx}}

\newcommand{\tpch}[1]{{\color{gray}(Related TPC-H choke point: #1)}}
