\chapter{Business Intelligence Workload}
\label{sec:bi}

The draft BI workload was published at the \mbox{GRADES-NDA} workshop at \mbox{SIGMOD 2018}~\cite{DBLP:conf/grades/SzarnyasPAMPKEB18}.
Version 1.0 of the workload is currently being prepared.

The LDBC SNB BI workload consists of two sets of operations:

\begin{itemize}
\item \textbf{Read queries.} Complex read queries touching a significant portion of the data. See \autoref{sec:bi-reads}.
\item \textbf{Microbatches of refresh operations.} A set of insert and delete operations, batched for a given time period (\eg an hour, a day, \etc). See \autoref{sec:bi-refreshes}.
\end{itemize}

\section{Benchmark scenario}
\label{sec:bi-benchmark-scenario}

\section{Parameter selection}
\label{sec:bi-paramgen}

During data generation, a sequence of \emph{substitution parameters} (\autoref{sec:substitution-parameters}) is generated.
Similarly to the Interactive workload, the parameter generation of the BI workload uses \emph{parameter curation}~\cite{DBLP:conf/tpctc/GubichevB14} to ensure that the query runtimes are  predictable (to some extent).

Several queries use multiple variants with different sets of input parameters.
\Eg for \queryRefCard{bi-read-14}{BI}{14}, 14(A) uses close countries while 14(B) uses countries that are far from each other.

In principle, query parameters are selected so that the query touches on a similar amounts of data.
For queries which are only constrained by one parameter, we select ranges in the distribution where the starting node has a similar amount of neighbours.
For example, if the query looks for \tMessages with a given \tTag:
(1) the Datagen computes the frequency of \tMessages per \tTags as a factor table,
(2) for each \tTag, we compute its distance (absolute difference) from a given percentile of the distribution is selected (\eg the \tTag on the 75th percentile),
(3) we pick the $k$ parameters with the lowest distance.

\section{Target metric}
\label{sec:bi-target-metric}

%%%%%%%%%%%%%%%%%%%%%%%%%%%%%%%%%%%%%%%%%%%%%%%%%%%%%%%%%%%%%%%%%%%%%%%%%%%%%%
%%%%%%%%%%%%%%%%%%%%%%%%%%%%%%%%%%%%%%%%%%%%%%%%%%%%%%%%%%%%%%%%%%%%%%%%%%%%%%
%%%%%%%%%%%%%%%%%%%%%%%%%%%%%%%%%%%%%%%%%%%%%%%%%%%%%%%%%%%%%%%%%%%%%%%%%%%%%%

\section{Reads}
\label{sec:bi-reads}

\input{query-cards/bi-read-01}
\input{query-cards/bi-read-03}
\input{query-cards/bi-read-04}
\input{query-cards/bi-read-05}
\input{query-cards/bi-read-06}
\input{query-cards/bi-read-07}
\input{query-cards/bi-read-08}
\input{query-cards/bi-read-10}
\input{query-cards/bi-read-14}
\input{query-cards/bi-read-16}
\input{query-cards/bi-read-17}
\input{query-cards/bi-read-18}
\input{query-cards/bi-read-21}
\input{query-cards/bi-read-22}
\input{query-cards/bi-read-25}

% reset counter to make sure the last query card isn't stuck in highlighted mode
\renewcommand{\currentQueryCard}{0}


%%%%%%%%%%%%%%%%%%%%%%%%%%%%%%%%%%%%%%%%%%%%%%%%%%%%%%%%%%%%%%%%%%%%%%%%%%%%%%
%%%%%%%%%%%%%%%%%%%%%%%%%%%%%%%%%%%%%%%%%%%%%%%%%%%%%%%%%%%%%%%%%%%%%%%%%%%%%%
%%%%%%%%%%%%%%%%%%%%%%%%%%%%%%%%%%%%%%%%%%%%%%%%%%%%%%%%%%%%%%%%%%%%%%%%%%%%%%

\section{Refreshes}
\label{sec:bi-refreshes}

\subsection{Inserts}
\label{sec:bi-inserts}

See \emph{Interactive Inserts} (\autoref{sec:interactive-inserts}).

\subsection{Deletes}
\label{sec:bi-deletes}

Each delete query removes

\begin{enumerate}
    \item a single edge between two existing nodes
    \item or a node, all incident edges and, in certain cases, nodes and edges that are transitively reachable on a certain path.
\end{enumerate}

\input{query-cards/bi-delete-01}
\input{query-cards/bi-delete-02}
\input{query-cards/bi-delete-03}
\input{query-cards/bi-delete-04}
\input{query-cards/bi-delete-05}
\input{query-cards/bi-delete-06}
\input{query-cards/bi-delete-07}
\input{query-cards/bi-delete-08}

