\chapter*{Definitions}

%{\flushleft \textbf{ACID}}: The transactional properties of Atomicity,
%Consistency, Isolation and Durability.
%
%
%{\flushleft \textbf{Commit:}} a control operation that:
%        \begin{itemize}
%            \item Is initiated by a unit of work (a Transaction) 
%            \item Is implemented by the DBMS 
%            \item Signifies that the unit of work has completed successfully
%and all tentatively modified data are to persist (until modified by some other
%operation or unit of work)  Upon successful completion of this control
%operation both the Transaction and the data are said to be Committed. 
%        \end{itemize}
%
%
%{\flushleft \textbf{DBMS:}} A Data Base Management System is a collection
%of programs that enable you to store, modify, and extract information from a
%database.
%
%
%
%{\flushleft \textbf{Durability:}} In general, state that persists
%across failures is said to be Durable and an implementation that ensures state
%persists across failures is said to provide Durability. In the context of the
%benchmark, Durability is more tightly defined as the SUT‘s ability to ensure
%all Committed data persist across any Single Point of Failure.
%
%{\flushleft \textbf{Measurement Window:}} This is the time window when the
%benchmark records statistics. It must fulfill the requirements defined in 
%\alert{Section XX}.
%
%{\flushleft \textbf{Performance Metric:}} The \ldbcsnb Reported Throughput as
%expressed in tps. This is known as the Performance Metric.
%
%{\flushleft \textbf{Price/Performance Metric:}} The \ldbcsnb total 3-year
%pricing divided by the Reported Throughput is price/tpsE. This is also known as
%the Price/Performance Metric.

{\flushleft \textbf{\ldbcsnb:}} The Linked Data Benchmark Council Social Network Benchmark. 


% {\flushleft \textbf{DBMS:}} A DataBase Management System. 

{\flushleft \textbf{System Under Test (SUT):}} This is the totality of the hardware and software that participates in a benchmark run, excluding parts that are exclusively used for driving the workload. If the parts driving the workload are collocated on the same operating system instance as the SUT, then this is also considered a part of the SUT. In client-server configurations where the test driver is not on a machine hosting any DBMS function the SUT is not considered to encompass the hardware or software which exclusively serves to drive the test workload.

{\flushleft \textbf{\datagen:}} This module is provided by LDBC SNB and produces the standard benchmark datasets to be loaded into the SUT for the benchmark. The data generation phase is not part of running the benchmark.

{\flushleft \textbf{Test Driver (Benchmark Driver):}} The test driver refers to the parts of the benchmark run that coordinate query execution and, if prescribed by a given benchmark, data loading.

{\flushleft \textbf{Workload (Benchmark):}} This is the totality of the tasks a particular benchmark does against an SUT. This includes data loading as well as the query/update workload. This does not include preparatory stages such as generating benchmark data with a data generator or transferring the data to the platform constituting the SUT.
The terms workload and benchmark are synonyms in this context. 

{\flushleft \textbf{Query Mix:}} The ratio of read and update queries
of a workload, and the frequency at which they are issued.

{\flushleft \textbf{Scale Factor (SF):}} The \ldbcsnb is designed to target systems of different size and scale. The scale factor determines the size of the data used to run the benchmark. The scale factor refers to the measured size of the data in Gigabytes when serialized in CsvBasic.

{\flushleft \textbf{Test Sponsor:}} The party which initiates an audit of a benchmark implementation over an SUT. This is typically the vendor of a key component of the SUT, \eg DBMS or hardware.
{\flushleft \textbf{Full Disclosure Report (FDR):}} It is a document which allows reproduction of any audited benchmark result by a third party. This contains complete description of the circumstances of the benchmark run, including version and configuration of SUT, dataset and test driver.

%{\flushleft \textbf{Test Run:}} The entire period of time during which Drivers
%submit and the SUT completes Transactions other than Trade-Cleanup.
%
%{\flushleft \textbf{Transaction:}} A Database Transaction is an ACID unit of work.

%{\flushleft \textbf{Valid Transaction:}} The term Valid Transaction refers to
%any Transaction for which input data has been sent in full by the Driver, whose
%processing has been successfully completed on the SUT and whose correct output
%data has been received in full by the Driver.
