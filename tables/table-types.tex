\begin{table}[h]
    \centering
    \begin{tabular}{|>{\typeCell}p{\attributeColumnWidth}|p{\largeDescriptionColumnWidth}|}
        \hline
        \tableHeaderFirst{Type} & \tableHeader{Description} \\
        \hline
        ID &  integer type with 64-bit precision. All IDs within a single entity type (\eg Person, Message) are unique, but different entity types (\eg a Forum and a Tag) might have the same ID.\\
        \hline
        32-bit Integer &  integer type with 32-bit precision\\
        \hline
        64-bit Integer &  integer type with 64-bit precision\\
        \hline
        32-bit Float &  integer type with 32-bit precision\\
        \hline
        64-bit Float &  integer type with 64-bit precision\\
        \hline
        String & variable length text of size 40 Unicode characters\\
        \hline
        Long String & variable length text of size 256 Unicode characters\\
        \hline
        Text &  variable length text of size 2000 Unicode characters\\
        \hline
        Date &  date with a precision of a day, encoded as a string with the following format: \texttt{yyyy-mm-dd}, where \texttt{yyyy} is a four-digit integer representing the year,
        the year, \texttt{mm} is a two-digit integer representing the month and \texttt{dd} is a two-digit integer representing the day. \\
        \hline
        DateTime &  date with a precision of milliseconds, encoded as a string with the following format: \texttt{yyyy-mm-ddTHH:MM:ss.sss+00:00}, where \texttt{yyyy} is a four-digit integer representing the year,
        the year, \texttt{mm} is a two-digit integer representing the month and \texttt{dd} is a two-digit integer representing the day, \texttt{HH} is a two-digit integer representing the hour, \texttt{MM} is a two
        digit integer representing the minute and \texttt{ss.sss} is a five digit fixed point real number representing the seconds up to millisecond precision. Finally, the \texttt{+00:00} of the end represents the
        timezone, which in this case is always GMT.\\
        \hline
        Boolean &  logical type, taking the value of either \texttt{True} of \texttt{False}\\
        \hline
    \end{tabular}
    \caption{Description of the data types. Some types such as 32-bit  Float and 64-bit Integer are currently not used in the benchmark.}
    \label{table:types}
\end{table}
