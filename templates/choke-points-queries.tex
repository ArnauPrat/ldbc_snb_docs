% Template-based generation of the table showing choke point coverage by queries.
% Note that 'update' queries (DEL, INS) and their choke points (CP-9.x) are omitted from the table
\begin{table}[htbp]
\scriptsize
\centering
\begin{tabular}{|l|
    
        |c|
    
} \hline

    
        & \chokePoint{ {{- choke_point -}} }
    
 \\ \hline


    {#- only list queries that have at least one choke point -#}
    
        \hline
        \queryRefCard{ {{- query.0 -}} }{
            IC
            BI
            
        }{ {{- query.3 -}} }
        
            
                &  \yes  
            
         \\ \hline
    


\end{tabular}
\caption{Coverage of choke points by queries.}
\label{tab:query_choke_point}
\end{table}
