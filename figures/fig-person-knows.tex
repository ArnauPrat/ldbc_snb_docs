\begin{figure}[ht]
  \centering
  \begin{subfigure}{\linewidth}
    \tikzstyle{vertex} = [circle, minimum width=45pt,draw,inner sep=2pt]
    \centering
    \begin{tikzpicture}[node distance=2cm]
      \node (v1) [vertex,xshift=0cm,yshift=0cm,fill=Person] {\small{$\instance{P_1}:\type{Person}$}};
      \node (v2) [vertex,xshift=5cm,yshift=0cm,fill=Person] {\small{$\instance{P_2}:\type{Person}$}};
      \node [below of=v1,yshift=1cm] {\tiny{\texttt{$\instc{P}{1}$: Feb 22 2010}}};
      \node [below of=v1,yshift=0.7cm] {\tiny{\texttt{$\instd{P}{1}$: Jul 26 2014}}};
      \node [below of=v2,yshift=1cm] {\tiny{\texttt{$\instc{P}{2}$: Mar 07 2010}}};
      \node [below of=v2,yshift=0.7cm] {\tiny{\texttt{$\instd{P}{2}$: Oct 17 2012}}};
      \draw[thick,->,>=stealth] (v1) --
      node [midway,above] {\small{$\instance{knows_{1,2}} : \type{Knows}$}}
      node [midway,below,align=right,text width=2.5cm] {\tiny{\texttt{$\instc{knows}{1,2}$: Dec 01 2011}}}
      node [midway,below,yshift=-0.3cm,align=right,text width=2.5cm] {\tiny{\texttt{$\instd{knows}{1,2}$: Jun 05 2012}}} (v2);
    \end{tikzpicture}
    \caption{An instance of a \tKnows edge connecting two \tPerson nodes. \emph{Creation} and \emph{deletion} dates are shown for each entity.}
    \label{fig:person-knows-graph}
    \end{subfigure}
  %
    \begin{subfigure}{\linewidth}
    \footnotesize
    \centering
    \begin{tikzpicture}[node distance=2cm,thick,every node/.style={transform shape}]
      %\tikzset{grid/.style={grey,very thin,opacity=1}} \draw[grid] (-3,0/1.5) grid (9,-6/1.5);

      \draw [thick,->,>=stealth] node [above,black] {} (-1,0/1.5) -- (6,0/1.5); % timeline
      \draw [grey,thin] (-0.5,0/1.5) node [above,black] {$\xSS$} -- (-0.5,-6/1.5);
      \draw [grey,thin]  (5.5,0/1.5) node [above,black] {$\xNC$} -- (5.5,-6/1.5);
      \draw [grey,thin]  (4.5,0/1.5) node [above,black] {$\xSE$} -- (4.5,-6/1.5);

      %P_1
      \draw[mark=*,mark size=2pt,mark options={color=green}] plot coordinates {(0.5,-1/1.5)} node [left] {$\instance{P_1}$}
      -- plot[mark=*,mark size=2pt,mark options={color=red}] coordinates {(5.0,-1/1.5)};

      %P_2
      \draw[mark=*,mark size=2pt,mark options={color=green}] plot coordinates {(1,-2/1.5)}  node [left] {$\instance{P_2}$}
      -- plot[mark=*,mark size=2pt,mark options={color=red}] coordinates {(4.0,-2/1.5)};

      % knows-c
      \draw[thin,grey,<->] plot coordinates {(1.3,-3/1.5)} node [left,black,xshift=-0.2cm] {\textcolor{grey}{$\instc{knows}{1,2}$}} node [align=right,xshift=-3.2cm,text width=2.55cm] {{\textcolor{green}{$\max(\instc{P}{1}, \instc{P}{2}) + \Delta$}}}
      -- plot[mark=*,mark size=2pt,mark options={color=blue}] coordinates {(2.5,-3/1.5)}
      -- plot coordinates {(4.0,-3/1.5)} node [xshift=3cm,text width=2.5cm] {{\textcolor{red}{$\min(\instd{P}{1}, \instd{P}{2}, \xSE)$}}};

     \draw [grey,thin,dashed] (4.0,-2/1.5) -- (4.0,-3/1.5);
     \shadedBox(1,-2/1.5,1/1.5);


     % knows-d
     \draw[thin,grey,<->] plot coordinates {(2.8,-4/1.5)} node [left,black,xshift=-0.2cm] {\textcolor{grey}{$\instd{knows}{1,2}$}} node [align=right,xshift=-4.7cm,text width=2.55cm] {{\textcolor{green}{$\instc{knows}{1,2} + \Delta$}}}
     -- plot[mark=*,mark options={color=blue}] coordinates {(3.5,-4/1.5)}
     -- plot coordinates {(4.0,-4/1.5)} node [xshift=3cm,text width=2.5cm] {{\textcolor{red}{$\min (\instd{P}{1},  \instd{P}{2})$}}}; %, \xNC

     \draw [grey,thin,dashed] (4.0,-3/1.5) -- (4.0,-4/1.5);
     \shadedBox(2.5,-3/1.5,1/1.5);

     % knows
      \draw[mark=*,mark size=2pt,mark options={color=green}] plot coordinates {(2.5,-5/1.5)}  node [left] {$\instance{knows_{1,2}}$}
      -- plot[mark=*,mark size=2pt,mark options={color=red}] coordinates {(3.5,-5/1.5)};
     \draw [thin,dashed,grey] (2.5,-4/1.5) -- (2.5,-5/1.5);
     \draw [thin,dashed,grey] (3.5,-4/1.5) -- (3.5,-5/1.5);

    \end{tikzpicture}
    \caption{
      Illustration of the intervals in which the \emph{creation} \textcolor{green}{$\bullet$} and \emph{deletion} \textcolor{red}{$\bullet$} dates can be selected.
      Thick black lines represent entity lifespans and thin grey lines represent valid intervals that dates can be selected in; \textcolor{blue}{$\bullet$} indicates the selected times (spanning the lifespan interval of the given entity).
      On the thin grey lines, thicker sections represent the minimal amount of time that must pass before selecting a value.
      In case of creation dates, this is used to ensure that the dependant entity exists for at least $\Delta$ time.
      In case of deletion dates, it is used to ensure that the entity exists for at least $\Delta$ time.
    }
    \label{fig:person-knows}
  \end{subfigure}

  \caption{Example graph and its intervals.}
  \label{fig:example-graph}
\end{figure}
