\chapter{Benchmark Specification}
\label{sec:benchmark-specification}

\section{Requirements}

\ldbcsnb is designed to be flexible and to have an affordable entry point. From small single node and in memory systems to large distributed multi-node clusters have its own place in \ldbcsnb.  Therefore, the requirements to fulfill for executing \ldbcsnb are limited to pure software requirements to be able to run the tools. All the software provided by \ldbcsnb have been developed and tested under Linux-based operating systems.

\ldbcsnb does not impose the usage of any specific type of system, as it targets systems of different nature and characteristics, from graph databases, graph processing frameworks and RDF systems, to traditional relational database management systems. Consequently, any language or API capable of expressing the proposed queries can be used. Similarly, data can be stored in the most convenient manner the test sponsor may decide, as long as it conforms with the execution rules. Finally, in order to have an official benchmark execution, the results will have to be audited and all the required information disclosed.

\section{Software and Useful Links}

\begin{itemize}
    \item \textbf{LDBC Driver -- \url{https://github.com/ldbc/ldbc_driver}}: The driver
    responsible for executing the \ldbcsnb workload.
    \item \textbf{\datagen{} -- \url{https://github.com/ldbc/ldbc_snb_datagen}}: The data
    generator used to generate the datasets of the benchmark.
\end{itemize}
