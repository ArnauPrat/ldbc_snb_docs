%%% DEVELOPMENT %%%

%%%%%%%%%%%%%%%%%%%%%%%%%%%%%%%%%%%%%%%%%%%%%%%%%%%%%%%%%%%%%%%%%%%%%%%%%%%%%%
%%%%%%%%%%%%%%%%%%%%%%%%%%%%%%%%%%%%%%%%%%%%%%%%%%%%%%%%%%%%%%%%%%%%%%%%%%%%%%
%%%%%%%%%%%%%%%%%%%%%%%%%%%%%%%%%%%%%%%%%%%%%%%%%%%%%%%%%%%%%%%%%%%%%%%%%%%%%%

\section{Participation of Industry and Academia}

%\alert{List the significant entities that have influenced the development of the
%benchmark. Explain what advice/feedback that provided and show how this information
%helped to shape the current form of the benchmark.}

The broader dissemination policies include public relations by the LDBC
board (\eg press releases), the periodic progress
statements of task forces, and the TUC internal
and external (public) information, disseminated via
portals. LDBC also organizes events, mainly TUC
meetings\footnote{http://ldbc.eu:8090/display/TUC/Events}
where LDBC members meet the LDBC
audience (IT practitioners, researchers, industry).
Two of such TUC meetings were held already in
November 2012 and April 2013, with a third TUC
meeting coinciding with Neo Technology's Graph-Connect in London (November 19, 2013).
Additionally, LDBC sponsors and co-organizes scientific
workshops. In 2013, it organized the 1st International
workshop on Benchmarking RDF Systems
(BeRSys), co-located with ESWC'2013; and the 1st
International workshop on Graph Data Management
Experiences and Systems (GRADES), co-located with
SIGMOD/PODS 2013, as well as the GraphLab
2013 workshop held in San Francisco, an event
focusing on graph programming frameworks. Finally,
LDBC members publish technical aspects of benchmark
development in scientific literature~\cite{angles2013benchmarking,boncz2013tpch,cattuto2013time,gubichev2013sparqling}.


%%%%%%%%%%%%%%%%%%%%%%%%%%%%%%%%%%%%%%%%%%%%%%%%%%%%%%%%%%%%%%%%%%%%%%%%%%%%%%
%%%%%%%%%%%%%%%%%%%%%%%%%%%%%%%%%%%%%%%%%%%%%%%%%%%%%%%%%%%%%%%%%%%%%%%%%%%%%%
%%%%%%%%%%%%%%%%%%%%%%%%%%%%%%%%%%%%%%%%%%%%%%%%%%%%%%%%%%%%%%%%%%%%%%%%%%%%%%

%\section{Changes of scope}

%\alert{When the scope of the benchmark was changed, either by increasing or decreasing
%the range of functionality tested, then details should be provided here with justifications.}

%%%%%%%%%%%%%%%%%%%%%%%%%%%%%%%%%%%%%%%%%%%%%%%%%%%%%%%%%%%%%%%%%%%%%%%%%%%%%%
%%%%%%%%%%%%%%%%%%%%%%%%%%%%%%%%%%%%%%%%%%%%%%%%%%%%%%%%%%%%%%%%%%%%%%%%%%%%%%
%%%%%%%%%%%%%%%%%%%%%%%%%%%%%%%%%%%%%%%%%%%%%%%%%%%%%%%%%%%%%%%%%%%%%%%%%%%%%%

%\section{Significant deviations}

%\alert{Provide detail on any change of direction during the development phase.}

%%%%%%%%%%%%%%%%%%%%%%%%%%%%%%%%%%%%%%%%%%%%%%%%%%%%%%%%%%%%%%%%%%%%%%%%%%%%%%
%%%%%%%%%%%%%%%%%%%%%%%%%%%%%%%%%%%%%%%%%%%%%%%%%%%%%%%%%%%%%%%%%%%%%%%%%%%%%%
%%%%%%%%%%%%%%%%%%%%%%%%%%%%%%%%%%%%%%%%%%%%%%%%%%%%%%%%%%%%%%%%%%%%%%%%%%%%%%
